\documentclass[12pt,a4paper]{article}
\usepackage{citegit}
\begin{document}
	\section*{Introduction}
	The \verb+citegit+ package can be used by a \LaTeX{} document to display
	(a limited amount of) information about the git repository that the document
	lives in. All functionality of the package should be considered highly
	unstable at the moment.
	\section*{Getting started}
	The package has been designed to require as little configuration as possible,
	so it only takes three steps to get up and running.
	\begin{enumerate}
		\item
		\label{step_package}
		Add the package to a document's preamble:
		\begin{verbatim}
			\usepackage{citegit}
		\end{verbatim}
		\item
		\label{step_cite}
		Cite the latest commit somewhere in the document:
		\begin{verbatim}
			\citegit{\githash}
		\end{verbatim}
		\item
		\label{step_compile}
		Compile the document:
		\begin{verbatim}
			pdflatex --shell-escape doc.tex
		\end{verbatim}
	\end{enumerate}
	Then, assuming nothing goes wrong for some reason, the output of step
	\ref{step_cite} will be something like \citegit{\githash}, the hash of the
	latest commit. If the repository is hosted on \href{https://github.com}{GitHub}
	then the displayed hash will automatically be a link to the appropriate page
	on GitHub.

%	This is a sample document that uses the \verb+citegit+ package.
%	It must be compiled with the \verb+--shell-escape+ option.
%	The package provides the following user commands.
%	\begin{description}
%		\item[\texttt{citegit}]
%		This command displays a reference to a snapshot of a file in the git
%		repository.
%		The length of the displayed commit hash is controlled by the \verb+hashes+
%		package option; the default hash length is \verb+[hashes=10]+.
%		\item[\texttt{githash}]
%		This command displays the full hash of the latest commit, currently \texttt{\githash}.
%		\item[\texttt{gitdate}]
%		This command displays the date and time of the latest commit.
%		There is currently no option to change the formatting because
%		the user can change it if they like in their \verb+git+ configuration.
%		\item[\texttt{gitauthor}]
%		This command displays the author of the latest commit, currently
%		\gitauthor.
%		\item[\texttt{gitmessage}]
%		This command displays the latest commit message, currently
%		``\gitmessage''
%	\end{description}
\end{document}
