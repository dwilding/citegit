\documentclass{article}
\usepackage[commits]{citegit}
\begin{document}
	This is a sample document that uses the \verb+citegit+ package.
	It must be compiled with the \verb+--shell-escape+ option.
	The package provides the following user commands.
	\begin{description}
		\item[\texttt{citegit}]
		This command displays a reference to a snapshot of a file in the git
		repository.
		To inspect the definition of this command, please see
		\citegit[23--60]{citegit.sty}{7920931825eb5c43502069d54033653512ebda7e}.
		The length of the displayed commit hash is controlled by the \verb+hashes+
		package option; the default hash length is \verb+[hashes=10]+.
		\item[\texttt{github}]
		This command sets the appropriate GitHub repository if the package did
		not do it automatically for some reason.
		In theory, at least, this command should never need to be used.
		\item[\texttt{githash}]
		This command displays the full hash of the latest commit, currently \texttt{\githash}.
		By default, `latest commit' means the latest commit that involves the file
		\verb+\jobname.tex+, but this can be changed by the \verb+commits+
		package option.
		For example, \verb+[commits=.]+ means the latest commit involving
		anything in the working directory, and just \verb+[commits]+ means the
		latest commit involving anything in the whole repository.
		\item[\texttt{gitdate}]
		This command displays the date and time of the latest commit.
		There is currently no option to change the formatting because
		the user can change it if they like in their \verb+git+ configuration.
		\item[\texttt{gitauthor}]
		This command displays the author of the latest commit, currently
		\gitauthor.
		\item[\texttt{gitmessage}]
		This command displays the latest commit message, currently
		``\gitmessage''
	\end{description}
\end{document}
