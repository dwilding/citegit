\documentclass[12pt,a4paper]{article}
\usepackage[dir=../]{citegit}
\begin{document}
	\section*{Introduction}
	The \verb+citegit+ package can be used by a \LaTeX{} document to display
	(a limited amount of) information about the git repository that the document
	lives in. All functionality of the package should be considered highly
	unstable at the moment.
	\section*{Getting started}
	The package has been designed to require as little configuration as possible,
	so it only takes three steps to get up and running.
	\begin{enumerate}
		\item
		\label{step_package}
		Add the package to a document's preamble:
		\begin{verbatim}
			\usepackage{citegit}
		\end{verbatim}
		\item
		\label{step_cite}
		Cite the latest commit somewhere in the document:
		\begin{verbatim}
			\citegit{\githash}
		\end{verbatim}
		\item
		\label{step_compile}
		Compile the document:
		\begin{verbatim}
			pdflatex --shell-escape yourdoc.tex
		\end{verbatim}
	\end{enumerate}
	Then, assuming nothing goes wrong, the output of step
	\ref{step_cite} will be something like \citegit{\githash}, the hash of the
	latest commit. The length of the hash can be changed by setting
	the \verb+[hashes=]+ package option, and if the repository is hosted on
	\href{https://github.com}{GitHub} then the hash will be a link to the
	appropriate GitHub page.

	\section*{Citing a file}
	The \verb+\citegit+ command has an optional argument that allows a particular
	file in the repository to be cited:
	\begin{verbatim}
		\citegit[yourfile]{fullhash}
	\end{verbatim}
	For example, this document first appeared as
	\citegit[../sample.tex]{64cfab6556112b5190a2deade78f0f628cf848a9}.
	There is yet another optional argument that allows a specific line, or a
	range of lines, of a particular file to be cited:
	\begin{verbatim}
		\citegit[start--end][yourfile]{fullhash}
	\end{verbatim}
	For example, at the time of writing the \verb+\citegit+ command
	was implemented as
	\citegit[99--157][../src/citegit.sty]{df2800ee183e86fc6338646d13d33e224045496a}.

	\section*{Information about the latest commit}
	The package has two modes of operation.
	In its default mode the package only considers commits that relate to
	\verb+yourdoc.tex+ and its dependencies.
	The dependencies of the document are determined in the following way.
	\begin{enumerate}
		\item
		If there is a makefile with target \verb+yourdoc.pdf+ then the dependencies
		will be lifted verbatim from the makefile. The makefile target that the
		package looks for can be changed by setting the \verb+[target=]+ package
		option in step \ref{step_package} above.
		\item
		Otherwise, the package looks through \verb+yourdoc.tex+ for occurrences
		of \verb+\input{*.tex}+ and \verb+\include{*.tex}+, and treats every
		filename it finds as a dependency.
		The document that the package searches can be changed by setting the
		\verb+[source=]+ package option.
	\end{enumerate}

	In its alternative mode of operation the package considers all commits that
	relate to the files in a given directory. This mode can be enabled by setting
	the \verb+[dir=]+ package option. For example, setting \verb+[dir=.]+ will
	cause the package to consider all commits relating to files in the same
	directory as \verb+yourdoc.tex+.

	Once the package has determined which commits to consider, it obtains the
	following information about the most recent one.
	\begin{description}
		\item[\texttt{\char`\\githash}]
		The full commit hash.
		\item[\texttt{\char`\\gitdate}]
		The commit date and time, currently ``\gitdate''. The display format depends
		on the configuration of \verb+git log+.
		\item[\texttt{\char`\\gitauthor}]
		The commit author, currently ``\gitauthor''.
		\item[\texttt{\char`\\gitmessage}]
		The commit message, currently ``\gitmessage'' \TeX{} command names are
		displayed in \texttt{teletype} font and issue numbers are linked to
		GitHub.
	\end{description}

	\section*{The state of the repository}
	The \verb+\ifclean+ command tests whether or not the repository is clean
	(no uncommitted changes), and can be used to display a message
	describing the state of the repository:
	\begin{verbatim}
		\ifclean{Repository is clean}{Repository is not clean}
	\end{verbatim}
	For example, the repository is currently \textbf{\ifclean{clean}{not clean}}.

	Note that (as above) \verb+\ifclean+ only considers \verb+yourdoc.tex+ and its
	dependencies, or files in the directory specified by the \verb+[dir=]+
	package option, when deciding whether or not the repository is clean.
	In this second case it may be desirable to ignore untracked files, in which
	case the \verb+\ifclean*+ command should be used instead.
\end{document}
